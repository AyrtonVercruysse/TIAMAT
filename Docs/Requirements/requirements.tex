\documentclass{article}
\title{Requirements Tiamat}
\author{Ayrton Vercruysse}
\begin{document}
\maketitle
\tableofcontents
\section{ASTs}
\subsection{Making ASTs}
\textbf{Description: } Every piece of code used in the IDE will be directly linked to an AST. For this matter there has to be a possibility to create new ASTs from scratch.
To create new AST trees we will need to implement the different parts of which an AST can exist. These different parts of an AST will be called a Node. We will start with
a list of possible nodes with which basic programs can be made. Later extra kinds of nodes can be added to the AST.
A superclass Node will be implemented which will organise the tree structure by providing operations like getParent(), which gets the parent of a Node, setChild(oldChild, newChild)
which will set a child of a node and getChilderen() to get the childeren of a Node. By using these operations new trees can be created. 
The types of nodes that will be implemented are: Placeholder, Begin, Block, Definition, Value, Argument List, Function, Function Call, Function Definition, Operation, Table, Table Call
Table Definition.
 \newline
\textbf{Priority:} High \newline
\textbf{Status: } \newline
\subsection{Deleting ASTs}
\textbf{Description: }When we have an excisting AST and parts of this AST became irreleveant there should be a possibility to remove this AST. The removing of an AST, in total,
or only a part of an AST will happen by replacing this AST by the Placeholder node. This possibility will be, as said in 1.1 implemented in the Node class, by means of the 
setChild(oldChild, newChild) function. \newline
\textbf{Priority:} High \newline
\textbf{Status: } \newline
\subsection{Replacing parts of ASTs}
\textbf{Description: }When new Nodes will be created it's childeren, if there are any, will be make by Placeholders. When making a function for example, we'll implement the operation,
but the two operands will be initialized with a Placeholder node. To be able to change these Placeholders to more relevant Nodes we should be able to 
replace existing Nodes by other Nodes, in this example maybe a Value, or an other Operation. This can be done on a similar way as the deletion of ASTs, by making use of the replaceChild(oldChild, newChild) method.\newline
\textbf{Priority:} High \newline
\textbf{Status: } \newline
\subsection{Rendering ASTs to text}
\textbf{Description: } To be able to pass the constructed code to the AmbientTalk interpreter we have to convert the AST to an understandable format for the interpreter. The 
interpreter works with a textfile containing AmbientTalk code. This means that the ASTs have to be converted to text. Converting an
AST to a text should be a function within each Node, calling the same function recursively on all of it's childeren.\newline
\textbf{Priority:} High \newline
\textbf{Status: } \newline
\section{Templates} 
\subsection{Templates creating ASTs}
\textbf{Description: } For convenience we will make use of templates. Templates will be frequently used parts of code that will be saved. The way templates will be stored in the 
memory will be by an AST. Every time a template is used withing the program a copy of this AST will be made and inserted into our current program.
These templates can be frequently used functions, but mainly these will be the ASTs representing basic elements of the language like the if-then-else sturcture, when-discovered,...\newline
\textbf{Priority:} High \newline
\textbf{Status: } \newline
\subsection{Creating templates from XML}
\textbf{Description: } To keep a clear view on what template functions we have, and make it easy for users to add templates, without having to know the implementation
of the entire program we will keep the Templates in a separate XML file. When having an XML file containing the structure to create new Templates a function will be implemented that creates, from the XML file, a new AST, which 
can be used as a Template within the program.\newline
\textbf{Priority:} High \newline
\textbf{Status: } \newline
\subsection{Create templates from functions}
\textbf{Description: } When a newly created function is often used by the user and he wants to save this function, the possibility to write this function, in correct XML, to
the existing XML file will be provided. This means that whenever the program is terminated functions can be stored inside the XML file.\newline
\textbf{Priority:} High \newline
\textbf{Status: } \newline
\section{Interface}
\subsection{Rendering of nodes}
\textbf{Description: }Each type of node has to have a screen representation. For each node we need a function that displays the Node on the screen, and 
that recursively calls the display function of all children of this node. We won't implement these displays within the separate types of nodes, discribed in 1.1, but 
for each node we will implement a separate display function. This with as goal that we can re-use our AST implementation without having to dead
with the program specific display possibilities.  \newline
\textbf{Priority:} High \newline
\textbf{Status: } \newline
\subsection{Lay-out engine}
\textbf{Description: } We will make a render manager, this manager will render all nodes and make sure they are organised on a ordered way. The render mananger will use
indetion to organise nodes across the screen.
\textbf{Priority:} High \newline
\textbf{Status: } \newline
\subsection{Creating functions on the spot}
\textbf{Description: } Whenever a new function (or variable) is created within the program a link to call this function will be added to the menu's. This means that we can
easily re-use a made variable or have an easy way to call an earlier created function. \newline
\textbf{Priority:} High \newline
\textbf{Status: } \newline
\subsection{Buttons}
\subsubsection{Menu's}
\textbf{Description: } To keep track of all templates or calls that can be made we will need a menu that gives us the possibility to insert templates in to the code.
We will use one a menu with some submenu's, devided for each kind of templates. These can be provided functions self implemented function, used variables,... \newline
\textbf{Priority:} High \newline
\textbf{Status: } \newline
\subsubsection{Run button}
\textbf{Description: } Thanks to requirements 4.2 we can evaluate made code with an external AmbientTalk app. To do this we'll add a button
to the main screen which will first make sure requirement 1.4 is done, the creation of a textfile with the current code, and later on run the AmbientTalk app, as described in requirement 4.2.
\textbf{Priority:} High \newline
\textbf{Status: } \newline
\subsubsection{Delete button}
\textbf{Description: } To be able to delete some code, as described in 1.2 we need a way to select what code will be deleted. Herefore we will
implement a delete button on the screen, which deletes the current selected piece of code. This can be done by pressing this button or by dragging the selected code to this place. 
\textbf{Priority:} High \newline
\textbf{Status: } \newline
\subsection{Views}
\subsubsection{Colorcoding keywords}
\textbf{Description: }As in many languages the use of colorcoding should be done here aswell. We  will try to achieve the same colorcoding as
being used in the official AmbientTalk IDE in Eclipse. \newline
\textbf{Priority:} Low \newline
\textbf{Status: } \newline
\subsubsection{Colorcoding types}
\textbf{Description: } Next to the colorcoding of keywords we will implement colorcoding on the types of words. This means that we will give
Placeholders different colors, just like VariableNames,... \newline
\textbf{Priority:} Low \newline
\textbf{Status: } \newline
\subsubsection{Codefolding}
\textbf{Description: }The use of AST should gives us the possibility to easily fold in certain parts (read piece of the AST) of our code. This can be done by just not
rendering certain nodes, and it's childeren, of the AST. \newline
\textbf{Priority:} Low \newline
\textbf{Status: } \newline
\subsubsection{Move code}
\textbf{Description: } By using the possibility of deleting parts of ASTs and saving parts of ASTs a copy-pasty system, or even moving of code to places that make sense belongs
to the possibilities.\newline
\textbf{Priority:} Low \newline
\textbf{Status: } \newline
\subsection{Gestures}
\subsubsection{Selection of code}
\textbf{Description: } To be able to replace, delete, move,... code we need a possibility to select certain parts of the code. This will be done
by a tap or doubletap gesture. The selected code will be marked and it will be possible to makes changes to this code.\newline
\textbf{Priority:} High \newline
\textbf{Status: } \newline
\subsubsection{Pinch-to-zoom}
\textbf{Description: }When starting on a piece of code by using the pinch-to-zoom function the selected part of code should be extended. This can be done by enlarging the selected
area from the current AST to the parent of this AST. \newline
\textbf{Priority:} High \newline
\textbf{Status: } \newline
\subsubsection{90° degrees turning for comments}
\textbf{Description: }When a piece of code is selected, and afterwards this piece of code gets turned around for 90 degrees, this piece of code should be commented out. 
Commented code should not be written to the textfile when this is called on this AST. \newline
\textbf{Priority:} Low \newline
\textbf{Status: } \newline
\subsubsection{Fast scrolldown}
\textbf{Description: }When we want to scroll down a longer piece of code we can use the scroll with two fingers getures to get us scroll faster. \newline
\textbf{Priority:} Low \newline
\textbf{Status: } \newline
\section{Evaluating code}
\subsection{Writing code to textfile}
\textbf{Description: }To make use of the AmbientTalk app we need a textfile with the AmbientTalk code. We can make use of the toText() function of each node to translate our
current AST to plain text, and afterwards we will need to write this text, into a textfile.  \newline
\textbf{Priority:} High \newline
\textbf{Status: } \newline
\subsection{Call external AmbientTalk app}
\textbf{Description: }The external AmbientTalk app, which makes use of the textfile from the previous - insert number- should be called. To get this app called we will implement
a button on the screen, which first gets the textfile created and afterwards calls the AmbientTalk app. \newline
\textbf{Priority:} High \newline
\textbf{Status: } \newline
\subsection{Return to TIAMAt after evaluating code}
\textbf{Description: } The AmbientTalk app will evaluate the program. After the evaluation of this program and returning the result we should get back the our application to
create the possibility to edit our current program\newline
\textbf{Priority:} High \newline
\textbf{Status: } \newline
\section{Features}
\subsection{Extra interface for comments}
\textbf{Description: } An extra interface can be created in which we can store extra comments. This interface should only be called on object which could get any use of extra comments.
 \newline
\textbf{Priority:} Low \newline
\textbf{Status: } \newline
\subsection{Speaking comments}
\textbf{Description: } A nice feature is to add spoken comments to our program. The possibility to add spoken comments will be integrated in the
extra interface for spoken comments, announced in requirement 5.1. There will be added extra buttons to record comments and play these comments.\newline
\textbf{Priority:} Low \newline
\textbf{Status: } \newline
\subsection{Selector for Java classes}
\textbf{Description: } Within the AmbientTalk language it's possible to make use of a Java class selector. Adding this feature to this implementation
will make sure that our project becomes a better alternative to the pc one.\newline
\textbf{Priority:} Low \newline
\textbf{Status: } \newline
\subsection{Multiple tabs}
\textbf{Description: } If we want to make bigger programs it would be easy to be able to have multiple tabs. Certainly when big programs exsist of multiple files. To make this possible we would need
to save multiple AST trees and implement a possibility to switch between the different tabs.\newline
\textbf{Priority:} Low \newline
\textbf{Status: }
\subsection{Saving files}
\textbf{Description: } As we will implement a write to XML to save new Template into an XML file we could use this procedure to save an entire program in XML, and later rebuild our AST and on this way 
reconstruct our program. \newline
\textbf{Priority:} Low \newline
\textbf{Status: }
\end{document}
